\documentclass[a4paper]{article}
\usepackage{pst-all}
\usepackage{graphics,eurosym,latexsym}
\usepackage{times}
\usepackage{listings}
\lstset{columns=fixed,basicstyle=\ttfamily,numbers=left,numberstyle=\tiny,stepnumber=5,breaklines=true}
\bibliographystyle{plain}
\oddsidemargin=0cm
\evensidemargin=0cm
\newcommand{\be}{\begin{enumerate}}
\newcommand{\ee}{\end{enumerate}}
\newcommand{\bi}{\begin{itemize}}
\newcommand{\ei}{\end{itemize}}
\newcommand{\I}{\item}
\newcommand{\ty}{\texttt}
\newcommand{\pim}{\hat{\pi}_{\rm m}}
\textwidth=16cm
\textheight=23cm

\begin{document}

\title{\ty{alfy}: ALignment-Free local homologY}
\author{Mirjana Domazet-Lo\v{s}o and Bernhard Haubold}
\date{\input{date.txt}\!\!, \input{version.txt}}
\maketitle

\section{Introduction}
\ty{alfy} is a program for comparing one or more query sequences to a
set of subject sequences and determining the closest homologue among
the subjects along each query \cite{dom11:ali}. In this document we
describe how to get started with \ty{alfy} and what input format it
accepts. This is followed by a Tutorial demonstrating central aspects
of \ty{alfy} when applied to HIV-1 genomes.

\section{Getting Started}
\ty{alfy} is written in C and is intended to work on any UNIX system
with a C compiler. However, please contact MDL at
\ty{Mirjana.Domazet-Loso@fer.hr} or BH at \ty{haubold@evolbio.mpg.de}
if you have any problems with the program.
\begin{itemize}
  \I Download \ty{alfy}
\begin{verbatim}
git clone https://github.com/evolbioinf/alfy
\end{verbatim}
\item Change into the newly created directory
\begin{verbatim}
cd alfy  
\end{verbatim}
and list its contents
\begin{verbatim}
ls
\end{verbatim}
\item Generate \ty{alfy}
\begin{verbatim}
make
\end{verbatim}
This creates two executables inside the working directory, \ty{alfy}
and \ty{alfy64}. \ty{alfy} is the 32 bit version of the program,
\ty{alfy64} its 64 bit version. The 32 bit version can analyze up to
$2^{29}\approx 5\times 10^8$ bp, while the 64 bit version can
theoretically analyze up to $2^{61}\approx 2\times 10^{18}$ bp. In
practice, the program is only limited by the available computer
memory. On the other hand, the 32 bit version is somewhat faster than
its 64 bit sibling and uses only half as much memory. We therefore
recommend that you use the 32 bit version for analyzes of up to
$5\times 10^8$ bp.  \I List options
\begin{verbatim}
./alfy -h
\end{verbatim}
\ei

\section{Input Files}
Input sequences need to be in FASTA format and can contain only characters
from the set $\{\ty{A}, \ty{C}, \ty{G}, \ty{T}\}$. The script
\ty{cleanSeq.awk} converts sequences to upper case and
removes all characters $\notin \{\ty{A}, \ty{C}, \ty{G}, \ty{T}\}$
from the sequences (but not from their headers):
\begin{verbatim}
awk -f scripts/cleanSeq.awk foo.fasta > foo2.fasta
\end{verbatim}

\section{Tutorial}
We assume that \ty{alfy} is in your path and you are in a working
directory that contains the data files from the directory \ty{data}
distributed with \ty{alfy}. It should also contain the script
\ty{quantifyGenotypes.awk} from the directory \ty{scripts} distributed
with \ty{alfy}.

\bi \I Annotate the query strain \ty{A+DQ083238} using 42 pure strains
as subject:
\begin{verbatim}
alfy -i A+DQ083238.fasta -j hiv42.fasta

>A+DQ083238
1     225   8.620536   A1+AF004885
226   495   25.146444  A1+AF069670
496   930   37.790272  G+AF084936
931   1155  6.579832   A1+AF069670
1156  1770  21.300528  C+U52953
1771  1875  16.382023  C+AF067155
1876  2175  11.652174  A1+AF069670
2176  2655  25.058455  A1+AF004885
2656  3285  23.612083  C+AF067155
3286  3585  29.481606  C+AY772699
3586  3825  14.728033  C+AF067155
3826  3915  15.022472  C+U52953
3916  4185  16.133829  C+AY772699
4186  4335  12.577181  C+U52953
4336  5025  23.910015  A1+AF004885
5026  5205  24.882681  G+AF061642
5206  5295  24.943821  A1+AF004885
5296  5415  10.008404  A1+AF069670
5416  5820  12.434650  A1+U51190
5821  6030  16.382023  A1+AF484509
6031  6225  8.395973   A1+AF069670
6226  6555  11.592705  A1+AF004885
6556  6705  7.747899   A2+AF286238
6706  6825  0.000000   A1+AF004885
6826  7140  11.314382  A1+AF484509
7141  7485  8.079027   A1+AF004885
7486  7755  18.382900  D+K03454
7756  8640  19.603571  A1+AF004885
8641  8805  1.378151   A1+U51190
8806  9330  8.914405   A1+AF004885
9331  9699  15.767802  A1+U51190
\end{verbatim}
This means that positions 1--225 of A+DQ083238 are most closely
related to strain A1+AF004885. Further, across that interval the
average shustring length of the shustrings induced by the ``winning''
subject sequence---A1+AF004885 in this case---is 8.6, which is a measure of the strength of
the homology signal across that interval.
\I To get a clearer view of the distribution of the various annotations, we can
summarize the output:
\begin{verbatim}
alfy -i A+DQ083238.fasta -j hiv42.fasta  | 
awk -f quantifyGenotypes.awk

>A+DQ083238
A1+AF004885  38.0
A1+AF069670  11.4
C+AF067155   10.1
A1+U51190    9.7
C+U52953     8.8
C+AY772699   5.9
A1+AF484509  5.4
G+AF084936   4.5
D+K03454     2.8
G+AF061642   1.9
A2+AF286238  1.5
\end{verbatim}
This indicates that 38.0\% of strain A+DQ083238 is most closely related
to strain A1+AF004885, 11.4\% to strain C+AF067155, etc. We can
further summarize the result by collapsing annotations for strains that
belong to the same group:
\begin{verbatim}
alfy -i A+DQ083238.fasta -j hiv42.fasta  | 
sed 's/+.*//' | 
awk -f quantifyGenotypes.awk

>A
A1  64.6
C   24.7
G   6.3
D   2.8
A2  1.5
\end{verbatim}
\I \ty{alfy} works in three steps:
\be
\I Construct enhanced suffix array from all query and
subject sequences;
\I determine ``winning'' subjects across contiguous intervals
along a query;
\I carry out sliding window analysis of these closest neighbors intervals to
find the final annotation of a given query.
\ee
The last step is sensitive to the length of the sliding window. By
default this is set to 300 bp; if we set it to 400 bp, the results
change somewhat, but not dramatically so:
\begin{verbatim}
alfy -w 400 -i A+DQ083238.fasta -j hiv42.fasta  | 
sed 's/+.*//' | 
awk -f quantifyGenotypes.awk

>A
A1  63.1
C   26.8
G   7.6
D   2.5
\end{verbatim}
\I We may only be interested in ``strong'' homology signals in our
analysis. These are defined as regions where the average shustring
length is greater than the maximum shustring length occurring by chance
alone. Regions with an average shustring length below the threshold value are
marked  \ty{nh} for ``no homology'':
\begin{verbatim}
alfy -i A+DQ083238.fasta -j hiv42.fasta -M

>A+DQ083238
1     225   8.620536   A1+AF004885
226   495   25.146444  A1+AF069670
496   930   37.790272  G+AF084936
931   1155  6.579832   nh
1156  1770  21.300528  C+U52953
1771  1875  16.382023  C+AF067155
1876  2175  11.652174  A1+AF069670
2176  2655  25.058455  A1+AF004885
2656  3285  23.612083  C+AF067155
3286  3585  29.481606  C+AY772699
3586  3825  14.728033  C+AF067155
3826  3915  15.022472  C+U52953
3916  4185  16.133829  C+AY772699
4186  4335  12.577181  C+U52953
4336  5025  23.910015  A1+AF004885
5026  5205  24.882681  G+AF061642
5206  5295  24.943821  A1+AF004885
5296  5415  9.539326   A1+AF069670
5416  5820  13.280936  A1+U51190
5821  6030  16.382023  A1+AF484509
6031  6225  8.395973   A1+AF069670
6226  6555  11.592705  A1+AF004885
6556  6705  7.747899   nh
6706  6825  0.000000   nh
6826  7125  11.475837  A1+AF484509
7126  7275  2.210084   nh
7276  7485  11.358851  A1+AF004885
7486  7755  18.382900  D+K03454
7756  8505  20.278667  A1+AF004885
8506  8640  11.820225  A1+AF004885
8641  8835  1.378151   nh
8836  9330  9.064439   A1+AF004885
9331  9699  15.767802  A1+U51190
\end{verbatim}
\I As before, we can summarize the annotations:
\begin{verbatim}
alfy -i A+DQ083238.fasta -j hiv42.fasta -M |
sed 's/+.*//' | 
awk -f quantifyGenotypes.awk

>A
A1  57.5
C   24.7
nh  8.7
G   6.3
D   2.8
\end{verbatim}
We find that the K annotation  is now missing altogether and 8.7\% of
the annotations are revealed as ``weak'' (\ty{nh}).
\I To further investigate what an \ty{nh} annotation means, we can
extract such a region and blast it against the 42 subject
sequences. This returns only weak homologues.
\begin{verbatim}
cutSeq -r 6556-6705 A+DQ083238.fasta > nh.fasta
blastn -query nh.fasta -subject hiv42.fasta -outfmt 6  
\end{verbatim}
\footnotesize
\begin{verbatim}
A+DQ083238  A1+AF069670  78.698  169  15  10  1  148  5770  5938  3.71e-21  93.5
A+DQ083238  B+AY173951   93.333  60   4   0   1  60   5903  5962  4.79e-20  89.8
A+DQ083238  J+AF082395   87.838  74   9   0   1  74   5877  5950  1.72e-19  87.9
A+DQ083238  K+AJ249239   88.710  62   7   0   1  62   5756  5817  3.73e-16  76.8
\end{verbatim}
\normalsize
\I The amount of weakly homologous regions is sensitive to the minimal
length of a recombining fragment. This quantity is set by the \ty{-f}
option with a default value of $\ty{-w}/4=75$. Increasing this value
leads to fewer regions diagnosed as ``non homologous''. For example,
if we double the value of \ty{-f} we get
\begin{verbatim}
alfy -f 150 -i A+DQ083238.fasta -j hiv42.fasta -M |
sed 's/+.*//' | 
awk -f quantifyGenotypes.awk

>A
A1  64.0
C   24.7
G   8.5
D   2.8
\end{verbatim}

\I The previous result is again sensitive to the parameter
settings. In this case the central quantity is the significance value
for the maximum shustring length. The mathematical theory for this was
derived in~\cite{hau09:est} and it is set with the \ty{-P}
option. This can be interpreted like a classical $P$-value: it is the
error probability when rejecting the null hypothesis that a shustring
of some given length is due to chance alone. The default value of the
\ty{-P} option is 0.4, which may strike you as rather high---why not
use the classical $P=0.05$ threshold? It is important to realize that
a query has to contain not just one shustring that is longer than the
maximum shustring length, but this property has to apply on average to
\textit{all} shustrings across a given window. If we set \ty{-P} to a
much more stringent value like 0.05, we get
\begin{verbatim}
alfy -i A+DQ083238.fasta -j hiv42.fasta -M -P 0.05 | 
sed 's/+.*//' | 
awk -f quantifyGenotypes.awk

>A
A1  44.2
nh  23.8
C   22.7
G   6.3
D   2.9
\end{verbatim}
The false negative rate is now misleadingly high (23.8\%).
\I Next, we explore the opposite problem of an inflated false positive
rate. The files \ty{s1.fasta} and \ty{s2.fasta} contain the two
simulated homologous sequences $S_1$ and $S_2$, respectively. The sequences are
separated by 100 mismatches per kb and have the homology structure
shown in Figure~\ref{fig:hom}: $S_2$ contains a gap between positions
40--60 kb, which is duly detected by \ty{alfy} as a region without homology when
running it with $S_1$ as query and $S_2$ as subject:
\begin{verbatim}
alfy -i s1.fasta -j s2.fasta -M

>S1
1      40125   13.227115  S2
40126  60045   9.835734   nh
60046  100000  13.444236  S2
\end{verbatim}
However, if we set \ty{-P} from its default value of 0.4 to, say, 0.5,
a semblance of homology across the 20 kb gap is created:
\begin{verbatim}
alfy -i s1.fasta -j s2.fasta  -M -P 0.5

>S1
1 100000 12.639876 S2
\end{verbatim}
\begin{figure}
\begin{center}
\begin{pspicture}(-0.4,-2)(10,1)
\psset{xunit=0.1cm}
\psaxes[Dx=10]{}(0,-1)(100,-1)
\psset{linewidth=15pt}
\psline[linecolor=darkgray](0,1)(40,1)
\psline[linecolor=lightgray](40,1)(60,1)
\psline[linecolor=darkgray](60,1)(100,1)

\psline[linecolor=darkgray](0,0)(40,0)
\psline[linecolor=darkgray](60,0)(100,0)
\psline[linecolor=black,linewidth=2pt,linestyle=dashed](40,0)(60,0)
\rput(-4,1){$S_1$}
\rput(-4,0){$S_2$}
\rput(50,-2){Position (kb)}
\end{pspicture}




\end{center}
\caption{Homology structure of the two simulated sequences $S_1$ and
  $S_2$. \textit{Dark gray}: homologous regions; \textit{light gray}:
  sequence without homology; \textit{dashed line}: gap.}\label{fig:hom}
\end{figure}

\I Theoretically, $P=0.5$ should result in exact equality between
the threshold length and the average length of random shustrings. Since
the theory strictly applies only in the limit of infinite sequence
length~\cite{hau09:est}, we observe a slight deviation from this
ideal, but $P=0.45$ will do the trick:
\begin{verbatim}
alfy -i s1.fasta -j s2.fasta -M -P 0.45

>S1
1      40125   13.227115  S2
40126  60045   9.835734   nh
60046  100000  13.444236  S2
\end{verbatim}
\I If you would like to simulate your own sequences to explore the
behavior of \ty{alfy}, here is the protocol for generating \ty{s1.fasta} and
\ty{s2.fasta}:
\bi
\I Create two homologous 100 kb sequences separated by 100 mismatches per
kb
\begin{verbatim}
ms 2 1 -s 1000 -r 0 100000 | ms2dna > tmp.fasta
\end{verbatim}
The program \ty{ms} is the coalescent simulator
written by Dick Hudson~\cite{hud02:gen}, which is available from his
web site. The program \ty{ms2dna} is available from
\[
\ty{https://github.com/evolbioinf/ms2dna}
\]
and the two programs \ty{getSeq} and
\ty{cutSeq} applied
in the next steps are part of the \ty{bioBox}
collection of bioinformatics tools available from
\[
\ty{http://github.com/evolbioinf/biobox}
\]
\I Extract sequence S1 and save it in file \ty{s1.fasta}
\begin{verbatim}
getSeq S1 tmp.fasta > s1.fasta
\end{verbatim}
\I Extract sequence S2 and cut two regions (\ty{-r}) separated by
20\,kb from it, which are joined (\ty{-j}) again to make the 20\,kb
deletion; the joined sequence is saved in file \ty{s2.fasta}.
\begin{verbatim}
getSeq S2 tmp.fasta | 
   cutSeq -j -r 1-40000,60001-100000 > s2.fasta
\end{verbatim}
\ei
\ei

\section{Change Log}
\be
\I Version 1.1 (February 4, 2011)
\bi
\I First stable version.
\ei
\I Version 1.2 (February 6, 2011)
\bi
\I Changed default value and the interpretation of \ty{-P}.
\I Changed bracketed output that indicates ``no homology'' to \ty{nh}.
\I Changed positions as starting from 0 to starting from 1.
\I Worked on documentation.
\ei
\I Version 1.3 (March 11, 2011)
\bi
\I Improved documentation.
\ei
\I Version 1.4 (December 8, 2011)
\bi
\I Fixed crash caused by contigs that are shorter than the default
window length (\ty{-w}).
\I Increased default value of minimum recombining fragment (\ty{-f}).
\ei
\I Version 1.5 (June 29, 2012)
\bi
\I The linker didn't accept \ty{-lm} at the beginning of the library
list in \ty{Src/Alfy/Makefile}, so moved it to the end of the list.
\ei
\I Version 1.6 (July 5, 2025)
\bi
\I Fixed double declaration of \ty{Shallow\_limit} in \ty{shallow.c}
and \ty{globals.c} that prevented compilation.
\I Ported to github.
\ei
\ee

\section{Acknowledgement}
This software is based on the \ty{dss\_sort} library by G. 
Manzini~\cite{man02:eng}.

\bibliography{ref}
%% Generate alfyDoc.bib:
%% bibtool -x alfyDoc.aux -o alfyDoc.bib
%%\bibliography{alfyDoc}
\end{document}

